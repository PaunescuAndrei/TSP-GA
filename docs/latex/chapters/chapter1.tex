\chapter{Traveling Salesman Problem}

The travelling salesman problem (TSP) is the challenge of finding the shortest possible route that visits every city from a list of cities exactly once and returns to the begining. It is a well-known algorithmic problem in the fields of computer science and operations research.
\par
TSP has gathered so much attention because it's very easy to describe yet very difficult to solve. It is an NP-hard problem in combinatorial optimization. The complexity of calculating the best route will increase when you add more destinations to the problem. 

\section{Input}

The problem receives as input a list of cities and their coordinates. The input received by the algorithm will be as a file in the TSPLIB\cite{tsplib} format.  

\section{Output}

The solution to the problem consists of an ordered list containing all the cities from the input that describes the shortest path found.
The exported output will be as a file in the TSPLIB\cite{tsplib} format.

